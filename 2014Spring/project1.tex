\documentclass[12pt]{article}
\usepackage{hyperlatex}

%\usepackage{enumitem}
\usepackage{ulem}
\usepackage[compact,medium,sc]{titlesec}
\usepackage{booktabs}
\usepackage[margin=0.5in]{geometry}
\usepackage[pdftex]{graphicx}
\usepackage{multirow}
\usepackage{pslatex}
\usepackage{color}

% paragraph style

\setlength\parskip{10pt}
\setlength\parindent{0in}

\W\newcommand{\sf}{\bf}

\htmltitle{CS100: Project 1}
\htmldepth{1}
\htmladdress{\xlink{lusth@cs.ua.edu}{mailto: lusth@cs.ua.edu}}
\htmlcss{lusth.css}

\title{CS100 Honors: Project 1\\
\date{Revision Date: \today}}

\begin{document}

\maketitle

\thispagestyle{empty}

\W\subsubsection*{\xlink{Printable Version}{project1.pdf}}
\W\htmlrule

\section*{Preamble}

This is your second C assignment.  You may develop your code anywhere,
but you must ensure it runs correctly when compiled by {\it gcc}
on a Linux or Unix system
before submission.

You should read up to and include the textbook chapter
\xlink{Input with Loops}{../book/index_16.html}
before starting this project.

%Spring 2009, 2014

\section*{Words!} 

\begin{quote}
``When I use a word,'' Humpty Dumpty said in rather a scornful tone,
``it means just what I choose it to mean -- neither more nor less.''
-- 
Lewis Carroll, in
{\it Through the Looking Glass}
\end{quote}

Create a directory named {\it project1} that hangs off of your 
{\it cs100} directory. Do your work in that directory.

Your task is to read in the entire corpus of Shakespeare's Plays and Sonnets
and print out the number of times a particular word occurs.

The corpus can be obtained with this linux command: 

\begin{verbatim}
    wget troll.cs.ua.edu/cs100/projects/shakespeare.txt
\end{verbatim}

\section*{Input / output}

The words of interest
are to be passed as command-line arguments, as in:

\begin{verbatim}
    $ gcc -Wall -g -o words words.c scanner.c shakes.c
    $ words xxxx yyyy zzzz
\end{verbatim}

where {\it xxxx}, {\it yyyy}, and {\it zzzz}
are the words that are to be counted.

The output of the program should be written to the console.

For example, the command:

\begin{verbatim}
    words romeo horse alas
\end{verbatim}

should print out how many times the words ``romeo'',
``horse'', and ``alas'' appear
in the corpus.

\section*{What are words?}

To obtain words from a file of text ({\it shakespeare.tex} is a file of text),
one {\it tokenizes} the file. Tokenizing a file means repeatedly reading
tokens from the file, where each token is composed of 
contiguous, non-whitespace characters.
Suppose the file contained the line:

\begin{verbatim}
    "O Romeo, Romeo!  wherefore art thou Romeo?"
\end{verbatim}

Tokenizing this file would yield the following tokens:

\begin{verbatim}
    "O
    Romeo,
    Romeo!
    wherefore
    are
    thou
    Romeo"?
\end{verbatim}

To turn these tokens into words, two things must happen.
First, any punctuation marks must
be removed. You can do this using the {\it filter} pattern; here is
the template for destructively filtering a string token:

\begin{verbatim}
    j = 0;
    for (i = 0; i < strlen(token); ++i)
        {
        if (???)  //test token[i] to see if it is a letter
            {
            token[j++] = token[i]
            }
        }
    token[j] = '\0';
\end{verbatim}

After removing punctuation, the tokens look like:

\begin{verbatim}
    O
    Romeo
    Romeo
    wherefore
    are
    thou
    Romeo
\end{verbatim}

The second thing that must be done is to reduce all the
capital letters to lower case letters.
This yields the following
list of words:

\begin{verbatim}
    o
    romeo
    romeo
    wherefore
    are
    thou
    romeo
\end{verbatim}

You can force a token to lowercase by destructively mapping the following
function over all the letters of the token:

\begin{verbatim}
    char
    toLower(char ch)   //return lowercase version of ch
        {
        if (ch <= 'Z')  //this is true if ch is uppercase
            return ch + 'a' - 'A';
        else            //otherwise ch is lowercase, so return it
            return ch;
        }
\end{verbatim}
            
\section*{Command-line arguments}

Here is a short C program that prints out all the command
line arguments:

\begin{verbatim}
    #include <stdio.h>

    int
    main(int argc,char **argv)
        {
        int i;

        // argv holds the name of the program plus the arguments
        // look at the length of argv, which is argc
        // loop through the arguments, skipping the name of the program

        printf("the name of the program is %s\n",argv[0]);
        for (i = 1; i < argc; ++i)
            {
            printf("     argument %d is %s\n",i,argv[i]);
            }

        return 0;
        }
\end{verbatim}

\section*{Reading from a file}

To read from a file, you are required to use the {\it scanner} module, which
you can retrieve with:

\begin{verbatim}
    wget http://troll.cs.ua.edu/cs100/projects/scanner.c
    wget http://troll.cs.ua.edu/cs100/projects/scanner.h
\end{verbatim}

Run this command in the same directory as your project files.  Here is a
small program that reads a file full of tokens
and outputs those tokens 
(one token per line) to the screen.

\begin{verbatim}
    #include <stdio.h>
    #include <stdlib.h>     //for exit
    #include "scanner.h"

    void readTokens(char *);

    int
    main(int argc,char **argv)
        {
        readTokens("someFileName");
        return 0;
        }

    void
    readTokens(char *fileName)
        {
        FILE *fp;
        char *token;
        
        fp = fopen(fileName,"r");
        if (fp == 0)
            {
            fprintf(stderr,"file %s could not be opened for reading\n",fileName);
            exit(1);
            }

        // output one token per line
        token = readToken(fp);
        while (token != 0)
            {
            printf("%s\n",token);
            free(token);
            token = readToken(fp);
            }

        // always close your open files when finished!!
        fclose(fp);
        }
\end{verbatim}

Note that the {\it readToken} function of the {\it scanner}
module will read
successively read sequences of contiguous, non-whitespace
characters as desired. When end-of-file is reached, {\it readtoken}
return the null pointer.

For your project, you will need to read the file twice. The first time,
you will count the number of tokens read. The second time, you will
store the tokens in an array you dynamically allocated. That array,
of course, will be sized according to the count of tokens.

\section*{Program Organization}

Your program should read and count the tokens. It should
then dynamically allocate the array to store the tokens and reread
the file. The second time the file is read,
the tokens are stored in the array. Next, it should
destructively filter the tokens in the array, removing punctuation.
Then it should destructively map the tokens, forcing the letters in each
token to lowercase.
Finally, for each word of interest
it should count the number of times it appears in the resulting word array.
Your {\it main} should look
something like this:

\begin{verbatim}
     int
     main(int argc,char **argv)
         {
         int count;

         // get the list of desired words
         ...

         // count the raw tokens
         count = fff("shakespeare.txt");

         // allocate the array
         tokens = ...

         // make sure the allocation succeeded
         ...

         // fill the array
         ggg("shakespeare.txt",tokens);
         
         // clean them up
         hhh(tokens);
         
         // take care of CAPITAL LETTERS
         iii(tokens);
         
         // loop through the desired words, counting the occurrences
         ...

         // free all the tokens
         ...

         return 0;
         }
\end{verbatim}

where {\it fff}, {\it ggg}, should be replaced by the most excellent function
names you will be using.

These are the functions you will need:

\begin{itemize}
\item
    a function ({\it fff} above)
    that tokenizes the input file, returning a count of the
    tokens read -- this function implements the counting pattern
\item
    a procedure that tokenizes the input file, filling an array of tokens
    ({\it ggg} above)
\item
    a procedure ({\it hhh} above)
    that receives an array of tokens -- the
    function loops through an array of tokens, removing punctuation
    from the tokens  -- this function
    implements the destructive map pattern using a helper function to deal 
    with an individual token
\item
    a procedure that receives a token and removes its punctuation --
    this function implements
    the destructive {\it filter} pattern.
\item
    a procedure ({\it iii} above)
    that loops through an array of tokens, forcing each
    of the tokens to lowercase --
    this function
    implements the destructive map pattern using a helper function to deal
    with an individual token
\item
    a procedure that receives a token and forces each of its characters
    to lowercase --
    this function implements the destructive {\it map} pattern.
\item
    a procedure that receives an array of desired words
    and a corpus (an array of
    words) -- the function loops through the array of desired words, calling
    a helper function to count the number of times a desired word appears
    in the corpus -- the function then prints the result for that word
\item
    a function that receives a word and a corpus and returns the number
    of times the work appears in the corpus -- this function implements
    the filtered-counting pattern
\end{itemize}

Place your {\it main} function in a 
file {\it words.c}. Place other functions/procedures
in a file named
{\it shakes.c}  and their function signatures into shakes.h.
Include {\it shakes.h} in {\it words.c}
with the command:

\begin{verbatim}
    #include "shakes.h"
\end{verbatim}

\section*{Stepwise refinement}

Never try to write an entire program in one go. Always implement
a little bit of what you need to do and test that bit thoroughly
with a specialized {\it main} function. This technique is known
as {\it stepwise} {\it refinement}.

\begin{description}
\item[Level 0]
Define a function that tokenizes a file, returning a count of
the number of tokens read.
For example, your specialized {\it main} for testing
this function might look something like this:

\begin{verbatim}
    int
    main(int argc,char **argv)
        {
        int count;
        count = fff("shakespeare.txt");
        printf("there are %d tokens\n",count);

        return 0;
        }
\end{verbatim}

where {\it fff} is the name of your token counting function.
From here on out, you may want to test using a smaller file instead of
the rather large {\it shakespeare.txt}.

\item[Level 1]
Define a function that tokenizes a file, filling a
correctly sized array with
the tokens read.
Test this function by looping through the array,
printing each token, one per line.
For example, your specialized {\it main} might look something like this:

\begin{verbatim}
    int
    main(int argc,char **argv)
        {
        int i;
        int count;
        char **tokens;

        count = fff("shakespeare.txt");

        //allocate the array
        tokens = ...
        
        //test that the allocation suceeded
        ...

        //fill the array
        ggg("shakespeare.txt",tokens);

        for (i = 0; i < count; ++i)
            {
            printf("%s\n",tokens[i]);
            }

        return 0;
        }
\end{verbatim}

where {\it ggg} is
the name of your array filling function.

\item[Level 2]
Define a function that when given a token (as a string), removes
the punctuation from the string.
For example, this function should
convert the string "fall," into the "fall".
The purpose of this function is to {\it clean} words.
For testing, your specialized {\it main} might look like:

\begin{verbatim}
    int
    main(int argc,char **argv)
        {
        char *token = strdup("fall,");
        print("the token was %s\n",token);
        jjj(token);
        print("the clean token is %s\n",token);
        return 0;
        }
\end{verbatim}

where {\it jjj} is replaced by the name of your cleaning function.
Test your function thoroughly before proceeding.

\item[Level 3]
Define a function that receives an array of tokens and cleans
all of the tokens.
This function uses the function
defined in Level 2.
Define a specialized {\it main} similar to that of Level 1.
Test your function thoroughly before proceeding.
\item[Level 4]
Define a function that when given a clean word forces all the
letters to lowercase.
Define a specialized {\it main} similar to that of Level 2 for
testing
your function thoroughly.
\item[Level 5]
Define a function that receives an array of cleaned tokens and decapitalizes
all of the.
This function uses the function
defined in Level 4.
Define a specialized {\it main} similar to that of Level 3.
Test your function thoroughly before proceeding.
\item[Level 6]
Define a function that receives a word and
a corpus (an array of lowercase words) and counts the number of
times the word appears in the corpus.
Test your function thoroughly with a specialized {\it main}.
\item[Level 7]
Define a procedure that receives an array of desired words
and a corpus
and prints each word and the number of times it appears in the
corpus. This function uses the function defined in Level 6.
This should be your final program.
\end{description}

\section*{Compliance Instructions}

To make sure that you have implemented your
program correctly,  this test:

\begin{verbatim}
    $words scooby doo
\end{verbatim}

should produce the following output:

\begin{verbatim}
    scooby appears 0 times
    doo appears 0 times
\end{verbatim}

You can check your other outputs on the forum.

\color{red}
If your code fails with a runtime error while running this test,
then you will
receive a zero for this assignment.
\color{black}

Programs submitted that generate compilation warnings will receive at
least a 5 point deduction.

Note that your answers do not have to be correct for your program to be graded,
only that it not fail.

\section*{Submission Instructions}

Change to the directory containing your assignment.  If you are working
on your USB-stick, run the command:

\begin{verbatim}
    submit cs100 XXXX project1
\end{verbatim}

Replace {\it XXXX} with your instructor name.

\section*{Due Date}

The due date for this assignment can be found on the class
\xlink{schedule}{../schedule.html}.

\end{document}
