\documentclass{article}  

\usepackage{booktabs}
\usepackage{hyperlatex}
\usepackage[margin=1.0in]{geometry}
\usepackage{color}

\htmldepth{1}
\htmltitle{CS 150}
\htmladdress{lusth@cs.ua.edu}
\htmlcss{lusth.css}

\T\setlength\parskip{6 pt}
\T\setlength\parindent{0 in}

\W\newcommand\sc\bf

\title{CS150 Honors: Project 3\\
\date{Revision Date: \today}}

\author{John C. Lusth}

\begin{document}

\maketitle

\thispagestyle{empty}

\W\subsubsection*{\xlink{Printable Version}{project3.pdf}}
\W\htmlrule

\section*{Preamble}

This is your last C programming assignment.  You may develop your
code anywhere, but you must ensure it runs correctly under Linux.
You should read up to and include the textbook chapter
``Two-dimensional Arrays'' before starting this project.

% Spring 2014, Spring 2010

\section*{Minesweeper!} 

\begin{quote}
``They blowed up good, blowed up {\it real} good'' --
{\it The Farm Film Report} with Big Jim McBob and Billy Sol Hurok
\end{quote}

Your task is to implement the classic game, {\it minesweeper}.  The game
consists of a board with the board divided into a grid.  Throughout the
grid, mines are scattered.  At the beginning of the game, all squares of
the grid are covered, obscuring the location of the mines.  The object
of the game is to uncover all the squares on the board that do not hold
mines, avoiding uncovering those squares that do. Squares are uncovered
one at a time.

When a square is uncovered, three things can happen. If the square holds
a mine, the mine blows up and the game ends. If the square does not hold
a mine, but is neighbors with a square that does, then the number of
neighbors that hold mines is shown in the uncovered square. If there are
no adjacent mines, then all neighboring squares are also uncovered. If
any of the newly uncovered squares are not adjacent to mines, then their
neighbors are uncovered, and so on.  Each square has eight neighbors,
unless it is on the edge of the board. In that case, it will have fewer
neighbors.

\section*{Graphical representation of a board}

You are to use Cheesy ASCII graphics (TM) to represent the board. Here
is a possible, but much too boring, representation of a board with five
rows and nine columns:

\begin{verbatim}
       0  1  2  3  4  5  6  7  8
    0 [-][-][-][-][-][-][-][-][-]
    1 [-][-][-][-][-][-][-][-][-]
    2 [-][-][-][-][-][-][-][-][-]
    3 [-][-][-][-][-][-][-][-][-]
    4 [-][-][-][-][-][-][-][-][-]
\end{verbatim}

In this representation, a  dash represents a covered square.  Suppose one
square with no adjacent mines is uncovered. In that case, the board
might look like:

\begin{verbatim}
       0  1  2  3  4  5  6  7  8
    0 [-][-][-][1][ ][ ][ ][ ][ ]
    1 [-][-][-][1][ ][ ][ ][ ][ ]
    2 [-][-][-][1][2][2][1][ ][ ]
    3 [-][-][-][-][-][-][2][1][1]
    4 [-][-][-][-][-][-][-][-][-]
\end{verbatim}

Be creative with your Cheesy ASCII drawings, especially when a mine
is uncovered.

\section*{Human-to-program communication}

The number of rows in grid, the number of columns in the grid, and the
number of mines are passed as command line arguments. The program then
randomly (or not) places the mines and prints out a representation of
the board with all squares covered.  It then goes into a loop, prompting
for a row and column of the first/next square to be uncovered. For each
square location entered by the user, the program displays the updated
board if the square did not hold a mine or an impressive pyrotechnic
display if it did.

\section*{Compliance Instructions}

Your program should run like this, but look better:

\begin{verbatim}
    $ minesweeper 5 9 5

    Here is the minefield:

       0  1  2  3  4  5  6  7  8
    0 [-][-][-][-][-][-][-][-][-]
    1 [-][-][-][-][-][-][-][-][-]
    2 [-][-][-][-][-][-][-][-][-]
    3 [-][-][-][-][-][-][-][-][-]
    4 [-][-][-][-][-][-][-][-][-]

    Which square do you wish to uncover?
    Enter a row and column: 1 7

       0  1  2  3  4  5  6  7  8
    0 [-][-][-][1][ ][ ][ ][ ][ ]
    1 [-][-][-][1][ ][ ][ ][ ][ ]
    2 [-][-][-][1][2][2][1][ ][ ]
    3 [-][-][-][-][-][-][2][1][1]
    4 [-][-][-][-][-][-][-][-][-]

    Which square do you wish to uncover?
    Enter a row and column:
\end{verbatim}

\section*{Submission Instructions}

Change to the directory containing your program
and run the command:

\begin{verbatim}
    submit cs150 YYY project3
\end{verbatim}

Replace \verb!YYY! with your instructor's name.

\section*{Due Date}

The due date for this assignment can be found on the class
\xlink{schedule}{../schedule.html}.

\end{document}
