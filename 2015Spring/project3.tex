\documentclass[12pt]{article}
\usepackage{hyperlatex}

%\usepackage{enumitem}
\usepackage{ulem}
\usepackage[compact,medium,sc]{titlesec}
\usepackage{booktabs}
\usepackage[margin=0.5in]{geometry}
\usepackage[pdftex]{graphicx}
\usepackage{multirow}
\usepackage{pslatex}
\usepackage{color}

% paragraph style

\setlength\parskip{10pt}
\setlength\parindent{0in}

\W\newcommand{\sf}{\bf}

\htmltitle{CS100: Project 3}
\htmldepth{1}
\htmladdress{\xlink{lusth@cs.ua.edu}{mailto: lusth@cs.ua.edu}}
\htmlcss{lusth.css}

\title{CS100: Project 3\\
\date{Revision Date: \today}}

\begin{document}

\maketitle

\thispagestyle{empty}

\W\subsubsection*{\xlink{Printable Version}{project3.pdf}}
\W\htmlrule

%Spring 2015

\section*{Preamble}

\begin{quote}
{\it A Picture is Worth a Thousand Words}
\end{quote}

Are you a visual person?  If you had a choice between a map and a set of written
directions, which would you use to find a new restaurant?  Would it have been
easier to code the last project if the flowchart had been replaced by a detailed
set of instructions on how to handle each case?

In the early days of the Internet, people constructed browsers that were entirely
text based (ask an old person if they remember {\tt Gopher}).  However, today's web
is filled with pictures.

Pictures can be stored in many formats.  You've probably heard of the more common
ones, such as {\tt jpeg} (from the Joint Photographic Experts Group)
and {\tt gif} (Graphics Interchange Format).
The vast majority of these formats store the image data in
a binary file.

There is one format, {\tt ppm}, that can use {\tt ASCII} files
to store an image.  These {\tt ppm} files consist of header information
and then a long string of numbers representing the {\it red}, {\it green} and
{\it blue} components of each pixel in the image.  It is not widely used, however.
Since the images are all stored in {\tt ASCII}, they are much larger than
using other (binary) formats.  Nevertheless, since these files contain {\tt ASCII}
data, they are a good format to use in introductory programming classes.

The {\tt ppm} format that is used for this project is shown below:

\begin{verbatim}
P3
width-in-pixels  height-in-pixels
maximum-color-value
pixel-1-1-red pixel-1-1-green pixel-1-1-blue  pixel-1-2-red pixel-1-2-green pixel-1-2-blue ...
pixel-2-1-red pixel-2-1-green pixel-2-1-blue  pixel-2-2-red pixel-2-2-green pixel-2-2-blue ...
...
pixel N-1-red pixel-N-1-green pixel-N-1-blue  pixel-N-2-red pixel-N-2-green pixel-N-2-blue ...
\end{verbatim}

The very first line in a {\tt ppm} file is always {\bf P3}.  After that, you have
three values (width, height, colors) that can be on a single line or separate lines.
Finally, you have the actual RGB values (three integers) for each pixel in the image.

So a very tiny {\tt ppm} file that is four pixels wide and 10 pixels tall, with the top
half red and the bottom half blue, would be represented by:

\begin{verbatim}
P3
4 10
255
255 0 0  255 0 0  255 0 0  255 0 0
255 0 0  255 0 0  255 0 0  255 0 0
255 0 0  255 0 0  255 0 0  255 0 0
255 0 0  255 0 0  255 0 0  255 0 0
255 0 0  255 0 0  255 0 0  255 0 0
0 0 255  0 0 255  0 0 255  0 0 255
0 0 255  0 0 255  0 0 255  0 0 255
0 0 255  0 0 255  0 0 255  0 0 255
0 0 255  0 0 255  0 0 255  0 0 255
0 0 255  0 0 255  0 0 255  0 0 255
\end{verbatim}

A {\tt ppm} file does not concern itself with the number of spaces (or tabs or newlines)
between the values in the file.

\section*{Before You Start}

This project does some basic manipulations on {\tt ppm} images.  Specifically,
your must must be able to swap (left to right), flip (top to bottom), rotate, and
invert a {\tt ppm} image.

A set of sample images can be retrieved from troll via:

\begin{verbatim}
wget troll.cs.ua.edu/cs100/foo.ppm
wget troll.cs.ua.edu/cs100/blocks.ppm
wget troll.cs.ua.edu/cs100/alabama.ppm
wget troll.cs.ua.edu/cs100/squares.ppm
wget troll.cs.ua.edu/cs100/cake.ppm
wget troll.cs.ua.edu/cs100/beach.ppm
wget troll.cs.ua.edu/cs100/eniac.ppm
wget troll.cs.ua.edu/cs100/mountain.ppm
wget troll.cs.ua.edu/cs100/bryant-denny.ppm
wget troll.cs.ua.edu/cs100/milky-way.ppm
wget troll.cs.ua.edu/cs100/river.ppm
wget troll.cs.ua.edu/cs100/earth.ppm
\end{verbatim}

Note on these pictures - they are listed from smallest to largest.
The first three (blocks and squares and alabama) are pretty small.
Cake is also not that big.
The last three images are large images.  Don't expect your program to
process them instantaneously - it will take several seconds.

In order to see if your program is working properly,
you need to be able to view these images.  Your system might (or might not)
have a viewer that supports {\tt ppm} images.  If it does not, then we
recommend you download the free program GIMP (GNU Image Manipulation Program).
You can find the download page for GIMP 
\xlink{here}{http://www.gimp.org/downloads/}.

Note: You can use GIMP to convert additional pictures into {\tt ppm} format
for testing if you wish.  Simply load an existing image that you have into
GIMP and then select the "Export As" option and a filetype of PPM image.  Make
sure to click {\tt ASCII} when asked how to export.  As part of this project,
you need to generate one new {\tt ppm} file.  It can be any image that you want.

Another Note:  When you use {\tt GIMP} to convert an image into
a {\tt ppm} file, it puts
a comment on the second line (between the {\tt P3} and the number of rows and
columns and colors.  You need to delete that comment line, as our program
does not handle comments in {\tt ppm} files.

\section*{Understanding this Project}

The following sub-sections are all designed to explain various aspects of
the project.  Make sure you fully understand all of these subsections before
you start coding the project.

\section*{Executing the Program}

All input to this program is given via the command line.  The last
command line argument should always be the {\tt ppm} image file.
The user can specify one or more manipulations for this file,
as shown below.  The four manipulations on a file are:
\begin{itemize}
\item {\tt -i} or {\tt -{}-invert}, which inverts all the pixels
in the image
\item {\tt -f} or {\tt -{}-flip}, which flips the image from top
to bottom
\item {\tt -s} or {\tt -{}-swap}, which swaps the image from left
to right
\item {\tt -r} or {\tt -{}-rotate}, which rotates the image
90 degrees clockwise
\item {\tt -o} or {\tt -{}-output}, which specifies an output
file name where the modified image should be written.  By
default, you should write the new image to {\tt stdout}.
\end{itemize}

Note that we are handling both styles of command line arguments
in this program, the single-dash and one-character argument, as
well as a double-dash and full-word argument.  In the Unix
world, the single-dash/one-character format was the standard
20 years ago, it is slowly being replaced by the double-dash/full-word
format.

Samples of program execution are shown below:

\begin{verbatim}
    ./image -f mountain.ppm
    ./image -f -i -r alabama.ppm
    ./image -s -o new.ppm squares.ppm
    ./image --swap --invert cake.ppm
    ./image --flip --output newFile.ppm mountain.ppm
\end{verbatim}

If the {\tt ppm} file specified as the
last command line argument actually exists (via the fact that
the {\it fopen} command does not fail), then you may assume that the
contents of the file represent a value {\tt ppm} image.

\subsection*{Establish a Pixel Struct and a PPM Struct}

You {\bf must} use the two {\tt struct}s shown below 
to store the {\tt ppm} image that you read from the input file.
These should be declared in {\tt ppm.h}.  The main program, 
{\tt image.c} will need this include file as it will need to know
about these two new types.

\begin{verbatim}
#ifndef PPM_STRUCT                          #ifndef RGB_STRUCT
#define PPM_STRUCT                          #define RGB_STRUCT

typedef struct ppm {                        typedef struct pixel {
    int rows;   // number of rows               int red;   // red value
    int cols;   // number of columns            int green; // green value
    int colors; // number of colors             int blue;  // blue value
    Pixel **pixels; // actual pixel data    } Pixel;
} ppmPic;
                                            #endif
#endif
\end{verbatim}

In the {\tt ppmPic} structure, the first three fields
should be fairly obvious.  The first and second, {\it rows} and
{\it cols}, are the number of rows and the number of columns in the file.
The third, {\it colors}, is the maximum color depth.
You get these values when you actually read the file data.

After this, you have {\tt Pixel **pixels}.  This represents the two-dimensional
array that holds all the R-G-B values for each pixel in the image.

It is {\bf critical} that you understand the {\tt pixels} part of a
{\tt ppmPic} struct.  This is a dynamically-allocated two-dimensional
array of integers that contain the RGB values for each pixel in the image.

In order to help you understand how these data structures work, we've
constructed a (primitive) 
\xlink{picture}{http://troll.cs.ua.edu/cs100/projects/project3picture.pdf}
that helps illustrate the use of these
data structures and also provided you a simple example of how a {\tt ppm} file
would be stored.  Please look at this
\xlink{picture}{http://troll.cs.ua.edu/cs100/projects/project3picture.pdf}
before reading any further.

\subsection*{An Example of How To Allocate Space and Initialize a PPM Struct}

Suppose you had the (very small) image shown below.  It is only 4 pixels wide
and six pixels tall. The top two rows are pure red, the middle two rows
are pure green, and the last two rows are pure blue.

\begin{verbatim}
P3
4 6
255
255 0 0  255 0 0  255 0 0  255 0 0
255 0 0  255 0 0  255 0 0  255 0 0
0 255 0  0 255 0  0 255 0  0 255 0 
0 255 0  0 255 0  0 255 0  0 255 0 
0 0 255  0 0 255  0 0 255  0 0 255
0 0 255  0 0 255  0 0 255  0 0 255
\end{verbatim}

To allocate space for this image, you need to allocate a new
{\tt ppmPic} object.  Once you have that object, you then read
the first few items (columns, rows, number of colors) and add
those values to the object.

\begin{verbatim}
    ppmPic *myPic = malloc( sizeof(ppmPic) );
    ...
    // read and ignore the P3 (we don't use it)
    // read the number of columns, assign it to the 'cols' field in myPic
    // read the number of rows, assign it to the 'rows' field in myPic
    // read the number of colors used, assign it to the 'colors' field in myPic
\end{verbatim}

Once you know the number of rows
and columns, you must then allocate space for the actual two-dimensional
array of pixels.  In the case above, we have six rows of Pixels and
four Pixel values on each row.  The first thing we do is allocate six
pointers to point to the first Pixel on each row.

\begin{verbatim}
    ...
    myPic->pixels = malloc(sizeof(Pixel *) * 6);
    ...
\end{verbatim}

Once you have these six pointers (pointing to the first Pixel in each
row), you then need to allocate space for the four Pixels on each row.

\begin{verbatim}
    ...
    myPic->pixels[i] = malloc(sizeof(Pixel) * 4);
    ...
\end{verbatim}

To reference an individual value within this two-dimensional array,
you can use either subscript notation or pointer notation.  The
notation {\tt data[a][b]} is equivalent to {\tt *(*(data+a)+b)}.

As an example of referencing individual elements using array notation for the case above,
where you have six rows with four Pixels on each row, consider the code below

\begin{verbatim}
    int val;
    ....
    // Assign 255 to the red field for the last Pixel on the second row (row=1 and col=3)
    myPic->pixels[1][3].red = 255;
    // Assign the green value for the second Pixel on the first row to val (row=0 and col=1)
    val = myPic->pixels[0][1].green;
    // Assign 0 to the blue field for the first Pixel on teh last row (row=5 and col=0)
    myPic->Pixels[5][0].blue = 0;
    ....
\end{verbatim}

\subsection*{Functions in your Program}

At a minimum, your {\tt ppm} module should have six functions, four for
manipulating the image and one that reads an image and one that writes an image.
The four manipulation functions are described in detail in the next section.

To handle reading a {\tt ppm} image, we recommend that you have a
{\tt readPic} routine within your {\tt ppm} module that:

\begin{itemize}
\item Takes as its single argument that is a {\tt FILE *} that points to the 
input {\tt ppm} file specified on the command line.
\item Allocates space for a new {\tt ppmPic} object.
\item Reads in values for columns, rows and number of colors in the image and
stores them in this newly allocated object.
\item Reads in the two-dimenionsal array of R-G-B values (Pixels) and stores
them in the dynamically allocated array {\tt pixels} within the {\tt ppmPic}
object.
\item Returns the newly allocated, and now fully populated with data, 
{\tt ppmPic} object that you declared and generated in this routine.
\item Note that you need to read the number of columns and rows in the
image from the file before you can do the dynamic allocation of the 
two-dimensional array {\tt pixels}.
\item Also note that each element in {\tt pixels} array requires you to read
three values, assigning the first to the {\tt .red} component, the second
to the {\tt .green} component, and the third to the {\tt .blue} component.
\end{itemize}

To handle writing a {\tt ppm} image, we recommend that you have a
{\tt writePic} routine within your {\tt ppm} module that:

\begin{itemize}
\item Takes two arguments, a {\tt ppmPic} structure and a {\tt FILE *} object.
\item The write routine simply transfers the information in the {\tt ppmPic}
structure to the specified file.
\item To write a file, print the string {\tt P3}, then
the number of columns in the file, the number of rows in the file, the number
of colors in each pixel, and finally the actual two-dimensional array of 
pixel data.
\item Note: printing the two-dimensional array of pixels is the vast majority of
the output (all but the first four tokens in the entire file).
\end{itemize}

You may introduce more functions if you wish.  As a general rule, none of the
functions you write should ever be more than 50 lines in length.  We encourage
short functions that perform a single task.  Long, complex functions are an
easy way to get stuck on projects.

\subsection*{What do the Various Operations Do?}

There are four manipulation options that you have to implement for this project.  Each
of them is described below.

\begin{itemize}
\item Invert - We recommend you implement this one first, it is the easiest
option.  For each pixel in the file, you simply {\it invert} the red and green
and blue values.  To invert a value, the new value is simply the number of
possible colors for that picture (a field within your struct) minus the current
value.  For example, if you have 255 colors in a picture and a pixel currently
as an RGB value of (200,100,50), then inverting it would yield
(255-200,255-100,255-50), or (55,155,205).
\item Swap - Swap is designed to swap the image from left to right.  Basically,
for a given row, the first pixel in that row is now the last pixel and the last
pixel in the row is now the first pixel.  For swap, you can simply move
from the top of the picture to the bottom, swapping the pixels left-to-right
and right-to-left, as you move down through the picture.  Remember that each
pixel has three values (red and green and blue), so you are swapping in
groups of three.
\item Flip - Flip is designed to flip the image from top to bottom.  The very
top row is now the bottom row, and the bottom row is now the top row.  So you
are moving entire rows of data up and down within the image.
\item Rotate - Rotate is designed to rotate the image 90 degrees clockwise.
As you do this, the number of rows and columns changes.  The number of rows
(height) of the original image is now the width of the new image.  The
number of columns (width) of the original image is now the height of the 
new image.  To implement this, you will need to allocate a new {\tt data}
block that stores the RGB values for all the pixels in the image and then
fill in the new {\tt data} block with values from the current image.  Once it is
fully populated, update the image object to point to this new block of data.
\end{itemize}

For each of these transformations, we recommend that you draw a simple picture
on a grid and then observe what happens if you flip the piece of paper upside
down, swap it left to right, or rotate it 90 degrees.  It will probably
help if you draw an image that is rectangular and not square.

\subsection*{Program Input}

As mentioned earlier, all input comes from the command line.  The last command
line argument represents the {\tt ppm} file and the other command line arguments
are some combination of {\tt -f}, {\tt -{}-flip},
{\tt -s}, {\tt -{}-swap}, {\tt -i}, {\tt -{}-invert}
and {\tt -r}, {\tt -{}-rotate}.
In addition, if the user specified {\tt -o} or {\tt -{}-output} then you should
write the new {\tt ppm} file to the argument specified immediately after the
{\tt -o} (or {\tt -{}-output}) argument.  If the user does not specify an output
file location, write the new file to standard output ({\tt stdout}).

If the user simply types the name of the executable, as in {\tt ./image} then 
you should print a short message explaining how to use the program.

You can use the scanner module to read the {\tt ppm} file contents into your
structure if you wish.

\subsection*{Program Output}

The output from this program is the modified {\tt ppm} file.  As
mentioned above, if the user specifies an output file name,
via either {\tt -o} or {\tt -{}-output}, then
write to that file.  If the user does not specify an output
file name via {\tt -o} or {\tt -{}-output}, then write to {\tt stdout}.

\subsection*{Where to Start}

The four image manipulation routines are all completely independent of each
other.  Given this, don't try and write the program all at once, as there is
no need to do that.  Instead, write it in smaller pieces.  One possible
development order is shown below:

\begin{itemize}
\item First, establish a directory in which you will work (project 3)
\item Name your main module {\tt image.c}.
If you want to use the CS 100 scanner routines, then you
will also need to copy the {\tt scanner.h} and
{\tt scanner.c} files from {\it troll} into this directory.
\item Create a module for the PPM image structure, 
{\tt ppm.c} and {\tt ppm.h}.
\item Define the basic data structures and be able to read in a {\tt ppm} image,
as in {\tt ./image foo.ppm}
\item Write the code for inverting an image (it is the easiest manipulation).
Be able to handle {\tt ./image -i foo.ppm} and {\tt ./image -{}-invert foo.ppm}.
\item Write the code that writes your image out to a file.  Until you do this,
you really need to test with very small {\tt ppm} images.  See the next section
for more information on testing with small images.
\item Write the code for swapping an image (left to right).  
\item Make sure your program can handle all combinations of inverting and swapping
the image.  That is, make sure it handles {\tt ./image -s -i foo.ppm} and
{\tt ./image -i -s foo.ppm}.
\item Write the code for flipping an image.
\item Make sure you can handle any combination of flipping and swapping and invert.
\item Write the code for rotating an image.
\item Make sure you can handle any combination of flipping, swapping, rotating and
inverting an image, either writing it to a file or to standard output.
\end{itemize}

\subsection*{Testing Your Program}

When you use real images, a {\tt ppm} file gets big quickly.  For example,
the {\tt mountain.ppm} file contains over 50 million characters.  This makes
debugging on real files problematic.  However, if you use really small files
then it is hard to see them using {\tt GIMP} or other image display software,
they are just too small.

One suggested option for debugging is to create a very small image, such
as the one shown below, in a file (use whatever name you want, such as
{\tt myTest.ppm}.

\begin{verbatim}
P3
4 9
255
200 0 0   150 0 0   100 0 0   50 0 0
200 0 0   150 0 0   100 0 0   50 0 0
200 0 0   150 0 0   100 0 0   50 0 0
0 200 0   0 150 0   0 100 0   0 50 0
0 200 0   0 150 0   0 100 0   0 50 0
0 200 0   0 150 0   0 100 0   0 50 0
0 0 200   0 0 150   0 0 100   0 0 50
0 0 200   0 0 150   0 0 100   0 0 50
0 0 200   0 0 150   0 0 100   0 0 50
\end{verbatim}

This image is small enough that you should be able to examine the output
your program generated for accuracy.  For example, the {\tt -s} or {\tt -{}-swap}
option should mean that your new first row is:

\begin{verbatim}
    50 0 0   100 0 0   150 0 0   200 0 0
\end{verbatim}

Likewise, the {\tt -i} or {\tt -{}-invert} option should give you a first row of:

\begin{verbatim}
    55 255 255   105 255 255   155 255 255   205 255 255
\end{verbatim}

Finally, the {\tt -f} or {\tt -{}-flip} option should give you a first row of:

\begin{verbatim}
    0 0 200   0 0 150   0 0 100   0 0 50
\end{verbatim}

Note, if you try and view {\tt foo.ppm} using {\tt GIMP} or other software,
you will need to increase the view to 400\% or 800\% or 1600\% of the actual size
(if you want to see it clearly).

We {\bf strongly} recommend that you test your program using a small image
such as the one described above.

\subsection*{A Comment on Debugging}

For this project, the new {\tt ppm} file that you generate goes to standard output,
{\tt stdout}, unless the user specifies otherwise.  To avoid having this output
mixed with any debugging print statements that you might generate, we suggest
sending all debugging output to {\tt stderr}.  Simply use the following
notation to write to {\tt stderr}.

\begin{verbatim}
    fprintf( stderr, "your debugging print statement\n", value(s)-to-print );
\end{verbatim}

As a reminder, {\tt stdin} is the default input stream, {\tt stdout} is the
default output stream, and {\tt stderr} is the default output stream for error
messages.  For example, when you attempt to compile a 
C program with errors, the errors are actually written to {\tt stderr}.

\subsection*{A Makefile is Required}

Please make sure to generate a {\tt makefile} for this project and include it in your
submission.  Since you have at least two modules in this project,
{\tt image.c} and {\tt ppm.c}, and possibly {\tt scanner.c},
you should include a {\tt makefile} that
allows you to compile this project cleanly and easily.

Your makefile should respond to the commands:

\begin{description}
\item[make] typing \verb!make! should have your makefile build your executable
\item[make clean] typing \verb!make clean! should remove the object files and the executable
\end{description}

Your makefile should never perform any unnecessary compilations.

\section*{Challenges}

Challenge: Add a {\tt -gray} option.  This option would create a grayscale
version of the image.  This version only has one value per pixel.  When
converting from color to grayscale, a good formula is
$val = 0.21R + 0.72G + 0.07B$ as humans are more sensitive to green.
In this case, write the new file to a {\tt .pgm} file instead of {\tt .ppm}.
The format for a {\tt .pgm} file is:

\begin{verbatim}
P2
width-in-pixels  height-in-pixels
maximum-gray-value
pixel-1-1 pixel-1-2 ... pixel-1-M
pixel-2-1 pixel-2-2 ... pixel-2-M
...
pixel-N-1 pixel-N-2 ... pixel-N-M
\end{verbatim}

\section*{Compliance Instructions}

First, check to make sure that you are using proper programming
style for this project.  The document
\xlink{style}{../style.html} contains a number of expected
conventions that must be followed for this project (as well as all
future projects).  Please make sure your program adheres to the
requirements identified in the 
\xlink{style guide}{../style.html}.

Second, create a file named {\it image.ppm} that contains a sample image, one
that you created, and used to test your program.

Make sure to test your makefile with the commands:

\begin{verbatim}
    $ make
    $ make clean
\end{verbatim}

\section*{Checklist}

Did you:

\begin{itemize}
\item
    name your main module {\it image.c}?
\item
    print an error message if given an invalid option (such as {\tt -change})?
\item
    place your functions in the proper modules?
\item
    comment your modules sufficiently?
\item
    test your makefile thoroughly?
\end{itemize}

Note: this list is not exhaustive! There may be other specifications
and requirements not listed here.

\section*{Submission Instructions}

Change to the {\it project3} directory containing your assignment.  Do an
{\it ls} command. You should see something like this:

{\color{red} image.c ~~ ~~ image.h ~~ ppm.c ~~ ppm.h
~~ makefile ~~ scanner.c ~~ scanner.h ~~ a PPM file that you created }

There is not much in {\tt image.h}, it is mainly just a series of
{\tt \#include} statements.  If you want to just put these statements
in {\tt image.c} and not have an {\tt image.h} file then that is
perfectly acceptable.

Extra files are OK, as long as you are in the correct directory. 
Submissions from the wrong directory
will be penalized.

Submit your program like this:

\begin{verbatim}
    submit cs100 YYY project3
\end{verbatim}

Remember to replace \verb!YYY! with your instructor name
(brown,cordes,lusth).

\section*{Due Date}

The due date for this assignment can be found on the class
\xlink{schedule}{../schedule.html}.

\end{document}
