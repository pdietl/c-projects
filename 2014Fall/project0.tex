\documentclass[12pt]{article}
\usepackage{hyperlatex}

%\usepackage{enumitem}
\usepackage{ulem}
\usepackage[compact,medium,sc]{titlesec}
\usepackage{booktabs}
\usepackage[margin=0.5in]{geometry}
\usepackage[pdftex]{graphicx}
\usepackage{multirow}
\usepackage{pslatex}
\usepackage{color}

% paragraph style

\setlength\parskip{10pt}
\setlength\parindent{0in}

\W\newcommand{\sf}{\bf}

\htmltitle{CS150: Project 0}
\htmldepth{1}
\htmladdress{\xlink{lusth@cs.ua.edu}{mailto: lusth@cs.ua.edu}}
\htmlcss{lusth.css}

\title{CS150: Project 0\\
\date{Revision Date: \today}}

\begin{document}

\maketitle

\thispagestyle{empty}

\W\subsubsection*{\xlink{Printable Version}{project0.pdf}}
\W\htmlrule

%Spring 2013

\section*{Preamble}

\begin{quote}
{\it Imitation is the sincerest form of flattery}
-- Aesop
\end{quote}

Real Computer Scientists abhor reinventing the wheel. 
So when given the task of writing a program, they search out and find programs
they feel are similar to the one they need to write.
They then modify the program, with permission,
so that it processes the input and
produces the output
they desire.
They accomplish this feat {\it even if they do not completely understand}
the original program upon which they are basing their new program.

Your task is to take an existing program that computes
baseball or softball on-base percentage 
and turn it into a program that computes
a soccer production rating for strikers or wingers.

\section*{Where to Start}

The first thing you should do is to setup a directory in which you will work.
Run the following commands in a terminal window, one at a time:

\begin{verbatim}
    cd
    cd cs150
    mkdir project0
    cd project0
\end{verbatim}

The first two commands move you into your {\it cs150} directory, while the next
two create and move into a project directory.
Now, 
retrieve the existing program for computing passer rating
with the command:

\begin{verbatim}
    wget troll.cs.ua.edu/cs150/projects/onbase.c
\end{verbatim}

Compile and run the program, using the following commands in a terminal
window:

\begin{verbatim}
    gcc -Wall -g onbase.c -o onbase 
    onbase
\end{verbatim}

In another terminal window, move into your project directory and run
the command:

\begin{verbatim}
    vim onbase.c
\end{verbatim}

This will allow you to look at the program while it is running in the
other window.
Try to figure out what part of the code does what by running the program
in the other window over and over.

\section*{How to Proceed}

Now, in the first window,
copy the existing program over into a new file named {\it soccer.c}
with the command:

\begin{verbatim}
    cp onbase.c soccer.c
\end{verbatim}

In the second window, quit {\it vim} and open up the new file, {\it soccer.c}.
Now begin modifying {\it soccer.c} so that it stops computing 
on-base percentages
and starts computing offensive production for soccer players.
Make one change at a time and test after each change.
Don't forget the {\it vim}'s undo button in case you introduce an error
into the code. Once a change is working, make sure you save the file.

Here is the 
formula for computing a soccer players offensive production rating:

\[
R = (5 G + 4 S + 3 C + 2 A + P) / T
\]

where
{\it G} is the number of goals,
{\it S} is the number of shots on goal,
{\it C} is the number of effective corner kicks,
{\it A} is the number of assists (primary),
{\it P} is the number of effective passes, and
{\it T} is the number of touches.

\section*{Compliance Instructions}

Compile your program, naming the executable {\it soccer}.

Create a file named {\it test.dat} that contains six numbers,
one per line,
for which your new program prompts.
Your program should prompt for the data in the same order as shown
in the description of the formula: goals, shots on goal,
corners, assists, passes, and touches.
Make sure there are no blank lines before, between, or after
the numbers.
If you have done this correctly, the command
(using an 'ell' not a 'one'):

\begin{verbatim}
    wc -l test.dat
\end{verbatim}
    
should produce the following output:

\begin{verbatim}
    6 test.dat
\end{verbatim}

Now, make sure that you have implemented your
program correctly by running this command:
    
\begin{verbatim}
    cat test.dat | soccer
\end{verbatim}

This method of running the program is called ``piping in the input from
a file''.
When you actually do this, the prompts your program makes for information
will all be strung together on a single line.
{\it Don't worry about it; it's a natural consequence of the way the
program was run.}

Your program should print out clearly the results
consistent with the six inputs.
\color{red}
If your code fails with a runtime error while running this test,
then you will receive a zero for this assignment.
\color{black}

Note that your answers do not have to be correct for your program to be graded,
only that the program does not crash. Of course, correct answers will yield 
a much higher grade.

Note also that a prerequisite for receiving any credit
is that your new program be based off the original program.

\section*{Challenges}

Try to get numbers to print like 1.324 instead of 1.3235294117647058.
You will need to learn about formatting numbers to do this challenge.

Try to get quantity agreement between numbers and their labels. For
example, the {\it soccer} program, when displaying the input data,
might output information this way:

\begin{verbatim}
    goals: 1
    shots on goal: 1
    effective corners: 5
\end{verbatim}

Instead, try outputting it this way:

\begin{verbatim}
     1 goal
     1 shot on goal
     5 effective corners
\end{verbatim}

Note the use of ``goal'' and ``shot'' instead of ''goals'' and 
``shots''.
You will need to learn about {\it if} statements to do this challenge.

\section*{Submission Instructions}

Change to the {\it project0} directory containing your assignment.  Do an
{\it ls} command. You should see something like this:

\begin{verbatim}
    soccer  soccer.c  onbase  onbase.c  test.dat
\end{verbatim}

Extra files are OK, as long as you are in the correct directory. 
Submissions from the wrong directory
will be penalized.

Submit your program like this:

\begin{verbatim}
    submit cs150 YYY project0
\end{verbatim}

Remember to replace \verb!YYY! with your instructor name
(eddy,lowen9,loewen1,lusth,williams9,williams1).
Note that {\it project0} ends in a zero, not a capital {\it Oh}.

\section*{Due Date}

The due date for this assignment can be found on the class
\xlink{schedule}{../schedule.html}.

\end{document}
