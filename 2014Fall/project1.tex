\documentclass[12pt]{article}
\usepackage{hyperlatex}

%\usepackage{enumitem}
\usepackage{ulem}
\usepackage[compact,medium,sc]{titlesec}
\usepackage{booktabs}
\usepackage[margin=0.5in]{geometry}
\usepackage[pdftex]{graphicx}
\usepackage{multirow}
\usepackage{pslatex}
\usepackage{color}

% paragraph style

\setlength\parskip{10pt}
\setlength\parindent{0in}

\W\newcommand{\sf}{\bf}

\htmltitle{CS100: Project 1}
\htmldepth{1}
\htmladdress{\xlink{lusth@cs.ua.edu}{mailto: lusth@cs.ua.edu}}
\htmlcss{lusth.css}

\title{CS100: Project 1\\
\date{Revision Date: \today}}

\begin{document}

\maketitle

\thispagestyle{empty}

\W\subsubsection*{\xlink{Printable Version}{project1.pdf}}
\W\htmlrule

\section*{Preamble}

This is your second C assignment.  You may develop your code anywhere,
but you must ensure it runs correctly when compiled by {\it gcc}
on a Linux or Unix system
before submission.

You should read up to and include the textbook chapter
\xlink{Input and Loops}{../book/index\_17.html}
before starting this project.

%Spring 2009, 2014

\section*{Words!} 

\begin{quote}
``When I use a word,'' Humpty Dumpty said in rather a scornful tone,
``it means just what I choose it to mean -- neither more nor less.''
-- 
Lewis Carroll, in
{\it Through the Looking Glass}
\end{quote}

Dr. Lusth gets quite irritated when he sees correspondences that
mix metaphors. For example, he goes apoplectic when he sees
someone use hashtags in a Facebook post\footnote{
Yes, he knows that you can search Facebook posts for a particular
hashtag, but just because you {\it can} doesn't mean you {\it should}.}.
Another major issue with him is the use of text message abbreviations/shortcuts
in more formal forms of messaging.  When he sees such a situation,
his eyes roll back in his head and he nearly passes out.
Your task is to help him out by writing a program to
translate messages that have
tokens suited only for text messages into their English equivalents.
For example, your program would translate the phrase:

\begin{verbatim}
    zomg ur such a nut rotfl kthxbye
\end{verbatim}

into:

\begin{verbatim}
    Golly gee willikers!
    You are such a nut.  I am literally rolling on the floor laughing.
    OK, thanks for your time. Toodle-pip!
\end{verbatim}

This is an incredibly difficult task. For example, how would
your program know if {\it ur} means {\it you} {\it are} or {\it your}? 
And how would your program know which words should be capitalized
and where punctuation would go?
To make your job easier, we will make some simplifications.

\section*{Starting the project}

Create a directory named {\it project1} that hangs off of your 
{\it cs100} directory. Do your work in that directory.

Your first task is to create a translation dictionary. For the example
above, a portion of the dictionary would look like:

\begin{verbatim}
    zomg    "Golly gee willikers!"
    ur      "you are"
    rotfl   "I am literally rolling on the floor laughing."
    kthxbye "OK, thanks for your time. Toodle-pip!"
    wtf     "What the frak!?!"
\end{verbatim}

Note that each text message token is paired with a string. Note
that there is no requirement that the string be on the same
line as the text token. For example, this rendering of
of a dictionary fragment would be legal as well:

\begin{verbatim}
    zomg
        "Golly gee willikers!"
    ur
        "you are"
\end{verbatim}

After creating the dictionary, you are ready to proceed to the
next step, tokenizing and storing the dictionary.

\section*{Tokenizing a file}

To obtain words from a file of text (i.e. your translation dictionary)
one {\it tokenizes} the file. Tokenizing a file means repeatedly reading
tokens from the file, where each token is composed of 
contiguous, non-whitespace characters or, if enclosed in quotes,
the entire set of characters contained within.
Suppose the file contained the lines:

\begin{verbatim}
    zomg    "Golly gee!"
    ur      "you are"
    rotfl   "I am literally on the floor laughing."
    kthxbye "OK, thanks for your time. Toodle-pip!"
    wtf     "What the frak!?!"
\end{verbatim}

Tokenizing this file would yield the following tokens:

\begin{verbatim}
    zomg
    Golly gee!
    ur
    you are
    rotfl
    I am literally on the floor laughing.
    kthxbye
    OK, thanks for your time. Toodle-pip!
    wtf
    What the frak!?!
\end{verbatim}

Thus, you would need a string array with room for 10 elements
to hold these tokens.

\subsection*{Counting the number of tokens in a file}

In order to allocate an array with the correct size,
you will need to read the dictionary file twice. The first time
you will count the number of tokens. A function that counts
the number of tokens will have the form:

\begin{verbatim}
    int
    countTokens(char *fileName)
        {
        //define and initialize the local variables
        ...
        //open the file
        ...
        //read the first token (use the scanner)
        ...
        while (!feof(???))    //check if the read is good
            {
            //the read was good, so increment the count
            ...
            //read the associated string (use the scanner)
            ...
            //increment the count again
            ...
            //free the token and the string (when counting only!)
            ...
            //read the next token (use the scanner)
            ...
            }
        //close the file
        ...
        //return the count
        ...
        }
\end{verbatim}

The function, of course, implements the counting pattern.

Once you have the count, you can dynamically allocate the string array.

\subsection*{Reading tokens into an array}

The procedure for reading in the dictionary tokens into an array
is similar to the {\it countTokens} function outlined in the previous
section. The major difference is in the body of the loop, which
will need to:

\begin{verbatim}
    //the read was good, so store the token in the array
    ...
    //increment the index
    ...
    //read the associated string
    ...
    //store the string in the array
    ...
    //increment the index again
    ...
    //rest of the loop body goes here
    ...
\end{verbatim}

Do not free the tokens and strings since they are needed!

\subsection*{Reading the message}

The next task is to read in the message to be translated.
This task is similar to reading in the dictionary, but the
function for counting and the procedure for filling the
array will be slightly different, as the message can be assumed
to contain no quoted strings, just tokens with no internal spaces.

\subsection*{Reading from a file}

To read from a file, you are required to use the {\it scanner} module, which
you can retrieve with:

\begin{verbatim}
    wget http://troll.cs.ua.edu/cs100/projects/scanner.c
    wget http://troll.cs.ua.edu/cs100/projects/scanner.h
\end{verbatim}

Run this command in the same directory as your project files.  Here is a
small program that reads a file full of tokens
and outputs those tokens 
(one token per line) to the screen.
It also returns a count of the tokens:

\begin{verbatim}
    #include <stdio.h>
    #include <stdlib.h>     //for exit
    #include "scanner.h"

    int readAllTokens(char *);

    int
    main(int argc,char **argv)
        {
        int c;
        c = readAllTokens("someFileName");
        printf("%d tokens printed\n",c);
        return 0;
        }

    int
    readAllTokens(char *fileName)
        {
        FILE *fp;
        char *token;
        int count = 0;
        
        fp = fopen(fileName,"r");
        if (fp == 0)
            {
            fprintf(stderr,"file %s could not be opened for reading\n",fileName);
            exit(1);
            }

        // output one token per line
        token = readToken(fp);
        while (!feof(fp))
            {
            printf("%s\n",token);
            ++count;
            free(token);
            token = readToken(fp);
            }

        // always close your open files when finished!!
        fclose(fp);

        return count;
        }
\end{verbatim}

Note that the {\it readToken} function of the {\it scanner}
module will read
successively read sequences of contiguous, non-whitespace
characters as desired. When end-of-file is reached, the {\it feof}
function returns true.

Again, you will need to read the dictionary file and the message
file twice. The first time,
you will count the number of tokens read. The second time you read, you will
store the tokens in an array you dynamically allocated. That array,
of course, will be sized according to the count of tokens.
You will free the tokens when counting, but will save the
in an array the second time the file is read.

\section*{Performing the translation}

Here you will need a loop to loop over the tokens in the
message. For each token, you will call a translating function that 
looks up the token in the dictionary. That function should
return zero if the token is not found in the dictionary.
If the token is found, then the function should return the
associated string.
Now, if the function returns zero, print the token itself,
since it must be a regular word. Otherwise, print what the
translating function returns.

The translating function loops/walks over the dictionary array, seeing if
there is a match between the given token and the current
token in the dictionary array. If there is a match, it immediately
returns with the associated string. Otherwise, if the loop
ends, then there must not have been a match and the function
should return zero to signify this result.

Remember, the tokens in the dictionary have even indices
and the associated strings have odd indices. Do not
compare tokens with the associated strings.

\section*{Program Organization}

Your main should read and count the dictionary tokens and strings,
via a helper function.
It should
then dynamically allocate the array to store the tokens and strings and reread
the file, again via helper functions.
Next, it should
read and store the message tokens, in a similar fashion, via yet more helper
functions.
Finally, it should translate the message, again via helper functions.

These are some of the support functions you will need:

\begin{itemize}
\item
    a function
    that tokenizes the dictionary file, returning a count of the
    tokens and strings read -- this function implements the counting pattern
\item
    a function that, given a count, allocates and returns an
    array of strings with the given size
\item
    a procedure that, given a file name and an array,
    tokenizes the dictionary file,
    filling the given array
\item
    a function
    that, given a file name,
    tokenizes the message file, returning a count of the
    tokens read -- this function implements the counting pattern
\item
    a procedure that, given a file name and an array,
    tokenizes the message file,
    filling the given array
\item
    a procedure
    that receives an array of message tokens, the number of message tokens, a dictionary array, and the length of the dictionary array -- the
    procedure loops through the array, calling a translating function --
    based upon the return value of the translating function,
    it prints the original token or its translation
\item
    a function that receives a token, a dictionary array, and the length of the dictionary array --
    the function loops through an dictionary array of tokens, comparing
    the tokens at the even indices with the given token --
    if there is a match, the string at the following index is returned --
    if no matches, zero is returned -- this function implements
    the search pattern
\end{itemize}

Place your {\it main} function in a 
file {\it translate.c}. Place other functions/procedures
in a file named
{\it support.c}  and their function signatures into {\it support.h}.
Remember to include {\it support.h} in {\it translate.c}
with the command:

\begin{verbatim}
    #include "support.h"
\end{verbatim}

\section*{Input and output}

As stated in the previous section,
your main function will go into the file {\it translate.c} and
the tokenizing and other supporting functions will go into
the file {\it support.c}.
You, therefore, would compile the translator with the command:

\begin{verbatim}
    $ gcc -Wall -g translate.c support.c scanner.c -o translate 
\end{verbatim}

Assuming your translation dictionary is in the file
{\it dictionary.txt}
and the message to be translated is in the file
{\it note.email},
you would translate the message with the command:

\begin{verbatim}
    $ translate dictionary.txt note.email
\end{verbatim}

The output of the program is the translation of the message
found in {\it note.email} and should be written to the console.

\subsection*{Command-line arguments}

The file names containing the dictionary and the message are
to be supplied as command-line arguments.
Here is a short C program that prints out all the command
line arguments:

\begin{verbatim}
    #include <stdio.h>

    int
    main(int argc,char **argv)
        {
        int i;

        // argv holds the name of the program plus the arguments
        // look at the length of argv, which is argc
        // loop through the arguments, skipping the name of the program

        printf("the name of the program is %s\n",argv[0]);
        for (i = 1; i < argc; ++i)
            {
            printf("     argument %d is %s\n",i,argv[i]);
            }

        return 0;
        }
\end{verbatim}

You should check to make sure there are exactly two command-line
arguments (beyond the program name).

\section*{Stepwise refinement}

Never try to write an entire program in one go. Always implement
a little bit of what you need to do and test that bit thoroughly
with a specialized {\it main} function. This technique is known
as {\it stepwise} {\it refinement}.

\begin{description}
\item[Level 0]
Define a function that tokenizes the dictionary file, returning a count of
the number of tokens and strings read.
For example, your specialized {\it main} for testing
this function might look something like this:

\begin{verbatim}
    #include <stdio.h>
    #include <stdlib.h>
    #include <string.h>
    #include "support.h"

    int
    main(int argc,char **argv)
        {
        int count;
        if (argc != 3)
            {
            printf("need two arguments!\n");
            exit(-1);
            }
        count = fff(argv[1]);
        printf("there are %d tokens and strings\n",count);

        return 0;
        }
\end{verbatim}

where {\it fff} is the name of your token counting function for
the dictionary file.
Place this {\it main} function in a file called {\it level0.c} (which should
include {\it support.h} and {\it scanner.h}).
The {\it fff} function (rename it, please!) should be placed in {\it support.c}
and its signature/prototype
in {\it support.h}.
Compile {\it level0.c} with the command:

\begin{verbatim}
    gcc -Wall -g level0.c support.c scanner.c -o level0
\end{verbatim}

Test it with the command:

\begin{verbatim}
    level0 dictionary.txt message.txt
\end{verbatim}

assuming your dictionary file is named {\it dictionary.txt} (the message file
is not used in {\it level 0}).

\item[Level 1]
Add a procedure that tokenizes a file, filling a
correctly sized array with
the tokens read.
Test this procedure by looping through the array,
printing each token, one per line.
For example, your specialized {\it main} might look something like this:

\begin{verbatim}
    #include <stdio.h>
    #include <stdlib.h>
    #include <string.h>
    #include "support.h"

    int
    main(int argc,char **argv)
        {
        int i;
        int count;
        char **dictionary;

        //as before in level 0
        ...

        //allocate the dictionary array
        dictionary = ggg(???);
        
        //fill the array
        hhh(argv[1],dictionary);

        printf("THE DICTIONARY...\n");
        for (i = 0; i < count; ++i)
            {
            printf("%s\n",dictionary[i]);
            }

        return 0;
        }
\end{verbatim}

where {\it ggg} is
the name of your array allocating function
and where {\it hhh} is
the name of your array filling function for the dictionary file.
As before, these functions go into {\it support.c}.
Create a \verb!level1.c! file to test your new functions.

\item[Level 2]
Add a function for counting the tokens in the message file,
whose name should be in argv[2]. Print out (and verify) the
count.

\item[Level 3]
Add a procedure that fills the message array. Your specialized
main should print out each token of the message.
It should also reuse your array allocation function.

\item[Level 4]
Add a function that when given a token and a dictionary array 
and size, returns either the translation/string or zero.
Test this function thoroughly by passing hard-wired tokens:

\begin{verbatim}
    translation = iii("ur",lolDict,lolDictSize);
    printf("ur is %s\n",translation);
    translation = iii("hello",lolDict,lolDictSize);
    printf("hello is %d\n",translation == 0);
\end{verbatim}

\item[Level 5]
Add the final procedure that loops through the
message array, calling the translation function and
printing the result appropriately.
Also, add a function that frees the tokens in the arrays and then
frees the arrays.

Place your {\it main} in {\it translate.c}, this time.

\end{description}

\section*{Compliance Instructions}

Create a dictionary file named {\it test.xlt} and, 
{\it at a minimum},
place in it the line:

\begin{verbatim}
    btw "By the way,"
\end{verbatim}

Create the message file {\it test.msg} and,
{\it at a minimum},
place in it the line:

\begin{verbatim}
    btw hello
\end{verbatim}

To make sure that you have implemented your
program correctly,  this test:

\begin{verbatim}
    $ translate test.xlt test.msg
\end{verbatim}

should produce the following output:

\begin{verbatim}
    By the way, hello
\end{verbatim}

\color{red}
If your code fails with a runtime error while running this test,
then you will
receive a zero for this assignment.
\color{black}

You may, of course, add to these files after you have tested the
minimum versions.

Programs submitted that generate compilation warnings will receive at
least a 5 point deduction.

Note that your answers do not have to be correct for your program to be graded,
only that it not fail.

\section*{Submission Instructions}

Change to the directory containing your assignment. 
Then run the command:

\begin{verbatim}
    submit cs100 YYYY project1
\end{verbatim}

Replace {\it YYYY} with your instructor name followed by your class time.

\section*{Due Date}

The due date for this assignment can be found on the class
\xlink{schedule}{../schedule.html}.

\end{document}
